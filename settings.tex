% لطفا هیچ تغییری در این فایل صورت ندهید.

\usepackage{subfig}
\usepackage{hyperref} % PDF links
\hypersetup{
	colorlinks = true, %Colours links instead of ugly boxes
	urlcolor = blue, %Colour for external hyperlinks
	linkcolor = blue, %Colour of internal links
	citecolor = red %Colour of citations
}
\usepackage[sanitizesort=false,sanitize={name=false},nomain,xindy,acronym,nonumberlist=true]{glossaries}
\usepackage[extrafootnotefeatures]{xepersian}
\usepackage[font={small},labelfont={normalsize,it}]{caption}
\settextfont[Scale=1.2]{B Nazanin}
\defpersianfont\minutesfont[Scale=1.2]{B Nazanin}
\setlatintextfont{Times New Roman}
\setdigitfont[Scale=1]{Yas}

\SepMark{-}
\newglossarystyle{fatoen}{%
	\renewenvironment{theglossary}{}{}
	\renewcommand*{\glsgroupskip}{\vskip \baselineskip}
	\renewcommand*{\glsgroupheading}[1]{\subsection*{\glsgetgrouptitle{##1}}}
	\renewcommand*{\glossentry}[2]{\noindent\glsentryname{##1}\dotfill\space \glsentrytext{##1}\par}
}
\newglossarystyle{entofa}{%
	\renewenvironment{theglossary}{}{}
	\renewcommand*{\glsgroupskip}{\vskip \baselineskip}
	\renewcommand*{\glsgroupheading}[1]{\begin{LTR} \subsection*{\lr{\glsgetgrouptitle{##1}}} \end{LTR}}
	\renewcommand*{\glossentry}[2]{\noindent\glsentrytext{##1}\dotfill\space \glsentryname{##1}\par}
}
\newglossarystyle{abbr}{%
	\renewenvironment{theglossary}{}{}
	\renewcommand*{\glsgroupskip}{\vskip \baselineskip}
	\renewcommand*{\glsgroupheading}[1]{}
	\renewcommand*{\glossentry}[2]{\noindent\glsentrytext{##1}\dotfill\space\Glsentrylong{##1}\par}
	\renewcommand*{\acronymname}{\hfill\rl{اختصارات و نمادها}\hfill}
}
\newglossary[glg]{english}{gls}{glo}{\rl{واژه‌نامه انگلیسی به فارسی}}
\newglossary[blg]{persian}{bls}{blo}{\rl{واژه‌نامه فارسی به انگلیسی}}
%\newglossary[glg]{english}{gls}{glo}{\hfill\rl{واژه‌نامه انگلیسی به فارسی}\hfill}
%\newglossary[blg]{persian}{bls}{blo}{\hfill\rl{واژه‌نامه فارسی به انگلیسی}\hfill}
\makeglossaries
\let\oldgls\gls
\let\oldglspl\glspl
\makeatletter
\renewrobustcmd*{\gls}{\@ifstar\@msgls\@mgls}
\newcommand*{\@mgls}[1] {\ifthenelse{\equal{\glsentrytype{#1}}{english}}{\oldgls{#1}\glsuseri{f-#1}}{\lr{\oldgls{#1}}}}
\newcommand*{\@msgls}[1]{\ifthenelse{\equal{\glsentrytype{#1}}{english}}{\glstext{#1}\glsuseri{f-#1}}{\lr{\glsentryname{#1}}}}
\renewrobustcmd*{\glspl}{\@ifstar\@msglspl\@mglspl}
\newcommand*{\@mglspl}[1] {\ifthenelse{\equal{\glsentrytype{#1}}{english}}{\oldglspl{#1}\glsuseri{f-#1}}{\oldglspl{#1}}}
\newcommand*{\@msglspl}[1]{\ifthenelse{\equal{\glsentrytype{#1}}{english}}{\glsplural{#1}\glsuseri{f-#1}}{\glsentryplural{#1}}}
\makeatother
\newcommand{\newword}[4]{
	\newglossaryentry{#1}     {type={english},name={\lr{#2}},plural={#4},text={#3},description={}}
	\newglossaryentry{f-#1} {type={persian},name={#3},text={\lr{#2}},description={}}
}
\defglsentryfmt[english]{\glsgenentryfmt\ifglsused{\glslabel}{}{\LTRfootnote{\glsentryname{\glslabel}}}}
\defglsentryfmt[acronym]{\glsentryname{\glslabel}\ifglsused{\glslabel}{}{\LTRfootnote{\glsentrydesc{\glslabel}}}}
\newcommand{\printabbreviation}{
	\cleardoublepage
	\pagestyle{empty}
	\phantomsection
	\baselineskip=.75cm
	\setglossarystyle{abbr}
	\begin{LTR}
		\Oldprintglossary[type=acronym]	
	\end{LTR}
	\clearpage
}
\newcommand{\printacronyms}{\printabbreviation}
\let\Oldprintglossary\printglossary
\renewcommand{\printglossary}{
	\let\appendix\relax
	\clearpage
	\phantomsection
	\addcontentsline{toc}{chapter}{واژه نامه انگلیسی به فارسی}
	\setglossarystyle{entofa}
	\Oldprintglossary[type=english]
	\clearpage
	\phantomsection
	\addcontentsline{toc}{chapter}{واژه نامه فارسی به انگلیسی}
	\setglossarystyle{fatoen}
	\Oldprintglossary[type=persian]
}
\newword{smallest}{\lr{Smallest Vertex Cover}}{\rl{کوچک‌ترین پوشش رأسی}}{\rl{}}
\newword{typeset}{\lr{Typesetting}}{\rl{حروف‌چینی}}{}
\newword{RandomVariable}{\lr{Random Variable}}{متغیر تصادفی}{متغیرهای تصادفی}
\newword{Action}{\lr{Action}}{کنش}{کنش‌ها} 
\newword{Optimization}{\lr{Optimization}}{بهینه‌سازی}{}
\newword{linked}{\lr{Linked List}}{\rl{لیست پیوندی}}{\rl{لیست‌های پیوندی}}
\newword{formatting}{\lr{Formatting}}{\rl{قالب‌بندی}}{}
\newword{file}{\lr{File}}{\rl{پرونده}}{\rl{پرونده‌ها}}
\newword{printer}{\lr{Printer}}{\rl{چاپگر}}{\rl{چاپگرها}}
\newword{site}{\lr{Web Site}}{\rl{وب‌گاه}}{\rl{وب‌گاه‌ها}}
\newword{flowchart}{\lr{Flow Chart}}{\rl{روندنما}}{\rl{روندنماها}}
\newword{inline}{\lr{Inline}}{\rl{داخلی}}{\rl{}}
\newword{lmatch}{\lr{Largest Matching}}{\rl{بزرگ‌ترین تطبیق}}{\rl{}}
\newword{bandwidth}{\lr{Bandwidth}}{\rl{پهنای‌باند}}{\rl{}}
\newword{download}{\lr{Download}}{\rl{دریافت}}{}

\newacronym{DFT}{\lr{DFT}}{\lr{Discrete Fourier Transform}}
\newacronym{QFT}{\lr{QFT}}{\lr{Quantum Fourier Transform}}
\newacronym[sort=alpha]{tensor}{\lr{$\otimes$}}{\rl{ضرب تانسوری}}
\newacronym{CDMA}{\lr{CDMA}}{\lr{Code Division Multiplexing Access}}
\newacronym{BAN}{\lr{BAN}}{\lr{Body Area Network}}
\newacronym{pdf}{\lr{PDF}}{\lr{Portable Document Format}}
\newacronym{gcd}{\rl{ب.م.م.}}{\rl{بزرگترین مقسوم‌علیه مشترک}}


\DefineVerbatimEnvironment{myverbatim}{Verbatim}{commandchars=+\[\]}
\DefineVerbatimEnvironment{mynewverbatim}{Verbatim}{commandchars=+|\$}

\makeatletter
	\newcommand*{\@thechapapp}{\@tartibi\c@chapter}
	\bidi@appto\appendix{\gdef\@thechapapp{\@harfi\c@chapter}}

	% ترتیبی کردن شماره فصل‌ها در فهرست مطالب در صورت استفاده از بسته hyperref
	\bidi@patchcmd{\Hy@org@chapter}{%
		\addcontentsline{toc}{chapter}%
		{\protect\numberline{\thechapter}#1}%
	}{%
		\addcontentsline{toc}{chapter}%
		{\protect\numberline{\@chapapp~\@thechapapp:}#1}%
	}{\typeout{We succeded in redefining \string\@chapter}}
	{\typeout{We failed in redefining \string\@chapter}}
	% اضافه کردن خط تیره بعد از شماره‌ فصل و بخش در متن
	%\renewcommand{\thesection}{\arabic{chapter}\@SepMark\arabic{section}\@SepMark}
	%\renewcommand{\thesubsection}{\arabic{chapter}\@SepMark\arabic{section}\@SepMark\arabic{subsection}\@SepMark}
	% اضافه کردن خط تیره بعد از شماره‌ها در فهرست مطالب و فهرست شکل ها و فهرست جدول ها
	\renewcommand{\cftsecaftersnum}{\@SepMark}%
	\renewcommand{\cftsubsecaftersnum}{\@SepMark}%
	\renewcommand{\cftfigaftersnum}{\@SepMark}%
	\renewcommand{\cfttabaftersnum}{\@SepMark}%
\makeatother

%%%%%%%%%%%%%%%%%%%%%%%%%%%%%%%
% زیاد کردن فاصله بین شماره‌ها و عنوان‌ها در فهرست مطالب
\setlength\cftchapnumwidth{4.5em}
\setlength\cftfignumwidth{3em}
\setlength\cfttabnumwidth{3em}
\setlength\cftsecnumwidth{2.5em}
\setlength\cftsubsecnumwidth{3.5em}
% زیاد کردن تورفتگی شماره‌ها و عنوان‌ها در فهرست مطالب
\setlength\cftsecindent{1.5em}
\setlength\cftsubsecindent{2.5em}

\renewcommand{\cfttoctitlefont}{\hspace*{\fill}\large\bfseries}
\renewcommand{\cftaftertoctitle}{\hspace*{\fill}}
\renewcommand{\cftlottitlefont}{\hspace*{\fill}\large\bfseries}
\renewcommand{\cftafterlottitle}{\hspace*{\fill}}
\renewcommand{\cftloftitlefont}{\hspace*{\fill}\large\bfseries}
\renewcommand{\cftafterloftitle}{\hspace*{\fill}}

%%%%%%%%%%%%%%%%%%%%%%%%%%%%%%%%%%%%%%%
% حروفی کردن شماره فصول و تغییر اندازه فونت فصول و بخش ها و زیر بخش ها
\titleformat{\chapter}[display]
  {\normalfont\large\bfseries}{\raggedright\chaptertitlename\ \tartibi{chapter}}{0pt}{\large\raggedright}
\titlespacing{\chapter}{2pc}{4cm}{1cm}[2pc]
\titleformat{\section}[block]{\bf{\normalfont\normalsize\bfseries}}{\thesection}{1em}{}
\titleformat{\subsection}[block]{\bf{\normalfont\small\bfseries}}{\thesubsection}{1em}{}
%%%%%%%%%%%%%%%%%%%%%%%%%%%%%%%%%%%%%%%
%حذف سطر اضافی بین دو مرجع متوالی در قسمت مراجع
\let\OLDthebibliography\thebibliography
\renewcommand\thebibliography[1]{
	\OLDthebibliography{#1}
	\setlength{\parskip}{0pt}
	\setlength{\itemsep}{0pt plus 0.3ex}
}
% ترتیبی کردن شماره فصل در سربرگ صفحات فرد
\renewcommand{\chaptermark}[1]{%
	\markboth{
		فصل \tartibi{chapter}~:~#1
	}{}
}
\makeatletter
	\bidi@patchcmd{\@harfi}{آ}{الف}
	{\typeout{Succeeded in changing `آ` into `الف`}}
	{\typeout{Failed in changing `آ` into `الف`}}
	\bidi@patchcmd{\@harfi}{ه}{هـ}
	{\typeout{Succeeded in changing `ه` into `هـ`}}
	{\typeout{Failed in changing `ه` into `هـ`}}
\makeatother

\let\oldAppendix\appendix
\renewcommand{\appendix}{
	\titleformat{\chapter}[display]
	{\normalfont\large\bfseries\flushleft}{\chaptertitlename\ \harfi{chapter}}{0pt}{\large}
	\oldAppendix
	\ifnofigures\else 
		\renewcommand{\thefigure}{\harfi{chapter}-\arabic{figure}} 
	\fi 
	\ifnotables\else 
		\renewcommand{\thetable}{\harfi{chapter}-\arabic{table}} 
	\fi
	\if@alg
		\renewcommand{\thealgorithm}{\harfi{chapter}-\arabic{algorithm}} 
	\fi
}

\setlength{\parskip}{0cm}

\renewcommand\citedash{\lr{--}}

