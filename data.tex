% % % % % % % % % % %
% نام دانشکده به همراه عنوان «دانشکده»
\faculty{دانشکده علوم‌ریاضی}
\facultyen{Mathematical Sciences}
% نام گروه آموزشی
\department{علوم کامپیوتر}
\departmenten{Computer Science}
% عنوان رشته‌ی تحصیلی
\subject{علوم کامپیوتر}
\subjecten{Computer Science}
% عنوان گرایش
\field{سیستم‌های هوشمند}
\fielden{Intelligent Systems}
% عنوان پایان‌نامه
\title{راهنمای استفاده از قالب \lr{\texttt{vruthesis}} جهت نگارش پایان‌نامه/رساله‌های دانشگاه ولی‌عصر (عج) رفسنجان}
\titleen{A Guide on how to use the \texttt{vruthesis} class to write Theses/Dissertations for Vali-e-Asr University of Rafsanjan}
\runtitle{عنوان کوتاه شده‌ی پایان‌نامه ...}
% نام و نام‌خانوادگی استاد راهنمای اول با پیشوند «دکتر»
\firstsupervisor{دکتر علی ...}
\firstsupervisoren{Dr. Ali ...}
% مرتبه‌ی علمی استاد راهنمای اول
\firstsupervisorrank{استادیار}
\firstsupervisorranken{Assistant Professor}
% اطلاعات استاد راهنمای دوم، در صورت وجود
\secondsupervisor{دکتر رضا ...}
\secondsupervisoren{Dr. Reza ...}
\secondsupervisorrank{استادیار}
\secondsupervisorranken{Assistant Professor}
% اطلاعات مشاور اول
\firstadvisor{دکتر محمد ...}
\firstadvisoren{Dr. Mohammad ...}
\firstadvisorrank{استادیار}
\firstadvisorranken{Assistant Professor}
% اطلاعات مشاور اول در صورت وجود
%\secondadvisor{دکتر حسین ...}
%\secondadvisoren{Dr. Hosein ...}
%\secondadvisorrank{استادیار}
%\secondadvisorranken{Assistant Professor}
% نام دانشجو
\name{علی}
\nameen{Ali}
% نام خانوادگی دانشجو
\surname{شکیبا}
\surnameen{Shakiba}
% ماه و سال دفاع - پس از ماه، عبارت «ماه» ذکر شود.
\thesisdate{آذر ماه ۱۳۹۶}
\thesisdateen{December 2017}
% تعداد واحد پایان‌نامه
\credit{۶}
\crediten{6}
% تاریخ دفاع
\defensedate{۱۳۹۶/۰۹/۳۰}
\defensedateen{December 21${}^{\text{st}}$, 2017}
% نمره‌ی دفاع
\grade{19.75}
\gradeen{19.75}
\letgrade{نوزده و هفتاد و پنج صدم}
\letgradeen{Ninteen and three quarters}
% درجه‌ی دفاع
\degree{عالی}
\degreeen{Excellent}
% اطلاعات داور داخلی اول
\firstinternalreferee{دکتر حسن ...}
\firstinternalrefereeen{Dr. Hasan ...}
\firstinternalrefereerank{استادیار}
\firstinternalrefereeranken{Assistant Professor}
% اطلاعات داور داخلی دوم
\secondinternalreferee{دکتر سجّاد ...}
\secondinternalrefereeen{Dr. Sajjad ...}
\secondinternalrefereerank{استادیار}
\secondinternalrefereeranken{Assistant Professor}
% اطلاعات داور خارجی اول
\firstexternalreferee{دکتر صادق ...}
\firstexternalrefereeen{Dr. Sadegh ...}
\firstexternalrefereerank{استادیار}
\firstexternalrefereeranken{Assistant Professor}
% اطلاعات داور خارجی دوم
\secondexternalreferee{دکتر باقر ...}
\secondexternalrefereeen{Dr. Bagher ...}
\secondexternalrefereerank{استادیار}
\secondexternalrefereeranken{Assistant Professor}
% اطلاعات ناظر تحصیلات تکمیلی
\viewer{دکتر کاظم ...}
\vieweren{Dr. Kazem ...}
\viewerrank{استادیار}
\viewerranken{Assistant Professor}
% متن تقدیم‌به
\totext{متن تقدیم به را در اینجا درج کنید}
% متن سپاسگزاری
\ack{متن سپاسگزاری را در اینجا درج کنید}
% متن چکیده فارسی
\abstractfa{چکیده‌ی خود را در اینجا قرار دهید}
% متن چکیده انگلیسی
\abstracten{Put your English abstract here.}
% واژگان کلیدی فارسی و انگلیسی که با «،» از هم جدا شده‌اند.
\keywordsfa{کلمه‌ی کلیدی اول، کلمه‌ی کلیدی دوم، کلمه‌ی کلیدی سوم.}
\keywordsen{First Keyword, Second Keyword, Third Keyword.}
% متن پیشگفتار
\preface{متن پیشگفتار را در اینجا قرار دهید.}
% % % % % % % % % % %
% این دستور را تغییر ندهید.
\vrutitle